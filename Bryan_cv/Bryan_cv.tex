%!TEX TS-program = xelatex
%!TEX encoding = UTF-8 Unicode
% Awesome CV LaTeX Template for CV/Resume
%
% This template has been downloaded from:
% https://github.com/posquit0/Awesome-CV
%
% Author:
% Claud D. Park <posquit0.bj@gmail.com>
% http://www.posquit0.com
%
%
% Adapted to be an Rmarkdown template by Mitchell O'Hara-Wild
% 23 November 2018
%
% Template license:
% CC BY-SA 4.0 (https://creativecommons.org/licenses/by-sa/4.0/)
%
%-------------------------------------------------------------------------------
% CONFIGURATIONS
%-------------------------------------------------------------------------------
% A4 paper size by default, use 'letterpaper' for US letter
\documentclass[11pt,a4paper,]{awesome-cv}

% Configure page margins with geometry
\usepackage{geometry}
\geometry{left=1.4cm, top=.8cm, right=1.4cm, bottom=1.8cm, footskip=.5cm}


% Specify the location of the included fonts
\fontdir[fonts/]

% Color for highlights
% Awesome Colors: awesome-emerald, awesome-skyblue, awesome-red, awesome-pink, awesome-orange
%                 awesome-nephritis, awesome-concrete, awesome-darknight

\definecolor{awesome}{HTML}{17202A}

% Colors for text
% Uncomment if you would like to specify your own color
% \definecolor{darktext}{HTML}{414141}
% \definecolor{text}{HTML}{333333}
% \definecolor{graytext}{HTML}{5D5D5D}
% \definecolor{lighttext}{HTML}{999999}

% Set false if you don't want to highlight section with awesome color
\setbool{acvSectionColorHighlight}{true}

% If you would like to change the social information separator from a pipe (|) to something else
\renewcommand{\acvHeaderSocialSep}{\quad\textbar\quad}

\def\endfirstpage{\newpage}

%-------------------------------------------------------------------------------
%	PERSONAL INFORMATION
%	Comment any of the lines below if they are not required
%-------------------------------------------------------------------------------
% Available options: circle|rectangle,edge/noedge,left/right

\photo{bryan2.png}
\name{Bryan Antony}{Quispe Ramos}

\position{Bachiller en Ingeniería Ambiental}
\address{Lima, Perú}

\mobile{+51 915228000}
\email{\href{mailto:bryan1qr@gmail.com}{\nolinkurl{bryan1qr@gmail.com}}}
\github{Bryan1qr}
\linkedin{bryan1qr}

% \gitlab{gitlab-id}
% \stackoverflow{SO-id}{SO-name}
% \skype{skype-id}
% \reddit{reddit-id}


\usepackage{booktabs}

\providecommand{\tightlist}{%
	\setlength{\itemsep}{0pt}\setlength{\parskip}{0pt}}

%------------------------------------------------------------------------------


\usepackage{float} \usepackage{multicol}

% Pandoc CSL macros
\newlength{\cslhangindent}
\setlength{\cslhangindent}{1.5em}
\newlength{\csllabelwidth}
\setlength{\csllabelwidth}{2em}
\newenvironment{CSLReferences}[3] % #1 hanging-ident, #2 entry spacing
 {% don't indent paragraphs
  \setlength{\parindent}{0pt}
  % turn on hanging indent if param 1 is 1
  \ifodd #1 \everypar{\setlength{\hangindent}{\cslhangindent}}\ignorespaces\fi
  % set entry spacing
  \ifnum #2 > 0
  \setlength{\parskip}{#2\baselineskip}
  \fi
 }%
 {}
\usepackage{calc}
\newcommand{\CSLBlock}[1]{#1\hfill\break}
\newcommand{\CSLLeftMargin}[1]{\parbox[t]{\csllabelwidth}{\honortitlestyle{#1}}}
\newcommand{\CSLRightInline}[1]{\parbox[t]{\linewidth - \csllabelwidth}{\honordatestyle{#1}}}
\newcommand{\CSLIndent}[1]{\hspace{\cslhangindent}#1}

\begin{document}

% Print the header with above personal informations
% Give optional argument to change alignment(C: center, L: left, R: right)
\makecvheader

% Print the footer with 3 arguments(<left>, <center>, <right>)
% Leave any of these blank if they are not needed
% 2019-02-14 Chris Umphlett - add flexibility to the document name in footer, rather than have it be static Curriculum Vitae


%-------------------------------------------------------------------------------
%	CV/RESUME CONTENT
%	Each section is imported separately, open each file in turn to modify content
%------------------------------------------------------------------------------



\paragraphstyle{Bachiller en Ingeniería Ambiental con conocimiento en Sistemas de Información Geográfica, Análisis de Datos Estadísticos, monitoreos ambientales, Análisis Geoespaciales usando lenguaje R. Tengo interés por el control de la contaminación, Investigación, Desarrollo e Innovación, así como cualquier actividad que que se desarrolle con responsabilidad ambiental.}

\hypertarget{experiencia-profesional}{%
\section{Experiencia profesional}\label{experiencia-profesional}}

\begin{cventries}
    \cventry{Consorcio Supervisor Ollachea}{Asistente técnico}{Puno, Perú}{mayo/2022-diciembre/222}{\begin{cvitems}
\item Apoyo en la realización de monitoreos de la calidad de aire, agua, ruido, realización de replanteos, verificación de trazos y niveles topográficos durante la ejecución de una obra vial.
\end{cvitems}}
    \cventry{Lenvir S.A.C.}{Asistente técnico}{Tacna, Perú}{agosto/2021 - abril/2022}{\begin{cvitems}
\item Apoyo en la elaboración de Instrumentos de Gestión Ambiental (FITSA, FTA y DIA) en los capítulos de linea base física, biológica y socioeconómica. Elaboración de mapas y planos.
\end{cvitems}}
    \cventry{Grupo Adrikap S.A.C.}{Asistente técnico}{Tacna, Perú}{marzo/2021 - julio/2021}{\begin{cvitems}
\item Apoyo en la elaboración de Instrumentos de Gestión Ambiental (DIA, PMRRSS ITSs), monitoreos ambientales de la calidad de aire, ruido, análisis e interpretación de datos estadísticos
\end{cvitems}}
\end{cventries}

\hypertarget{educaciuxf3n}{%
\section{Educación}\label{educaciuxf3n}}

\begin{cventries}
    \cventry{Bachiller en Ingeniería Ambiental}{Universidad Nacional Jorge Basadre Grohmann}{Tacna, Perú}{abril/2015-diciembre/2019}{\begin{cvitems}
\item Bachiller en Ciencias con mención en Ingeniería Ambiental, quinto superior.
\end{cvitems}}
    \cventry{Técnico en Ofimática}{Instituto de Informática y Telecomunicaciones}{Tacna, Perú}{marzo/2019 - junio/2019}{\begin{cvitems}
\item Uso de Microsoft Office a nivel avanzado, Instalación y mantenimiento de Sistemas Operativos
\end{cvitems}}
\end{cventries}

\hypertarget{programas}{%
\section{Programas}\label{programas}}

\begin{cventries}
    \cventry{Uso del software Quantum GIS a nivel avanzado}{QGIS}{■■■■■■■■■■■■■■■■■■■■■■■■}{ }{}\vspace{-4.0mm}
    \cventry{Uso del software ArcGIS a nivel avanzado}{ArcGIS}{■■■■■■■■■■■■■■■■■■■■■■■■}{ }{}\vspace{-4.0mm}
    \cventry{Uso del software AutoCAD a nivel avanzado}{AutoCAD}{■■■■■■■■■■■■■■■■■■■■■■■■}{ }{}\vspace{-4.0mm}
    \cventry{Uso del software RStudio a nivel avanzado}{RStudio}{■■■■■■■■■■■■■■■■■■■■■■■■}{ }{}\vspace{-4.0mm}
    \cventry{Uso del software Microsoft Excel a nivel intermedio}{Microsoft Excel}{■■■■■■■■■■■■■■■■□□□□□□□□}{ }{}\vspace{-4.0mm}
    \cventry{Uso del software Microsoft Word a nivel intermedio}{Microsoft Word}{■■■■■■■■■■■■■■■■□□□□□□□□}{ }{}\vspace{-4.0mm}
    \cventry{Uso del software Microsoft PowerPoint a nivel avanzado}{Microsoft PowerPoint}{■■■■■■■■■■■■■■■■■■■■■■■■}{ }{}\vspace{-4.0mm}
    \cventry{Uso del software Mendeley a nivel intermedio como gestor de referencias bibliográficas}{Mendeley}{ ■■■■■■■■■■■■■■■■□□□□□□□□}{ }{}\vspace{-4.0mm}
    \cventry{Uso del software SASPlanet a nivel avanzado}{SASPlanet}{■■■■■■■■■■■■■■■■■■■■■■■■}{ }{}\vspace{-4.0mm}
    \cventry{Uso del software Google Earth Pro a nivel avanzado}{Google Earth Pro}{■■■■■■■■■■■■■■■■■■■■■■■■}{ }{}\vspace{-4.0mm}
\end{cventries}

\hypertarget{cursos}{%
\section{Cursos}\label{cursos}}

\begin{cventries}
    \cventry{Mastergis}{Geoestadística aplicada al medioambiente}{Lima, Perú}{mayo/2022-junio/2022}{\begin{cvitems}
\item Curso de especialización en geoestadística usando el software ArcGIS
\end{cvitems}}
    \cventry{Mastergis}{Experto en SIG con QGIS}{Lima, Perú}{diciembre/2021 - abril 2022}{\begin{cvitems}
\item Programa profesional del software Quantum GIS.
\end{cvitems}}
    \cventry{Mastergis}{Variabilidad espacial de cultivos a través de ArcGIS y R}{Lima, Perú}{febrero/2022 - marzo/2022}{\begin{cvitems}
\item Curso de geoestadística aplicado a cultivos.
\end{cvitems}}
    \cventry{Mastergis}{Experto en SIG con ArcGIS}{Lima, Perú}{setiembre/2021 - febrero 2022}{\begin{cvitems}
\item Programa profesional del software ArcGIS.
\end{cvitems}}
    \cventry{Mastergis}{Google Earth Engine con R}{Lima, Perú}{diciembre/2021 - enero 2022}{\begin{cvitems}
\item Curso de programación con lenguaje R para tratamiento de datos espaciales de la nube de Google.
\end{cvitems}}
    \cventry{Mastergis}{Analisis Espacial con R}{Lima, Perú}{diciembre/2021 - enero 2022}{\begin{cvitems}
\item Curso de Análisis Espacial usando el lenguaje de programación con R.
\end{cvitems}}
    \cventry{CEIA-UNALM}{Análisis de Datos de Calidad de Aire con R y Rstudio}{Lima, Perú}{octubre/2021 - noviembre/2022}{\begin{cvitems}
\item Curso de programación con R y Rstudio aplicado a la Calidad de Aire.
\end{cvitems}}
    \cventry{ILCID}{Data Science: Estadística y Análisis de Datos en R}{Lima, Perú}{julio/2021-octubre/2021}{\begin{cvitems}
\item Programa de certificación especializado con honores.
\end{cvitems}}
    \cventry{Grupo Adrikap S.A.C.}{Elaboración de Instrumentos de Gestión Ambiental}{Tacna, Perú}{setiembre/2021-setiembre/2022}{\begin{cvitems}
\item Curso de elaboración de los diversos capítulos de un IGA.
\end{cvitems}}
    \cventry{Gestión Integral HQSE}{Especialista en AutoCAD}{Lima, Perú}{diciembre/2020-junio/2021}{\begin{cvitems}
\item Curso de especialización en AutoCAd en los niveles I, II y III.
\end{cvitems}}
    \cventry{Mastergis}{Estudio de Impacto Ambiental con ArcGIS}{Lima, Perú}{mayo/2021-junio/2021}{\begin{cvitems}
\item Curso de Elaboración mapas necesarios en un Estudio de Impacto Ambiental usando el software ArcGIS.
\end{cvitems}}
    \cventry{Gestión Integral HQSE}{Diploma en SIG}{Lima, Perú}{agosto2020-enero/2021}{\begin{cvitems}
\item Diploma en Sistemas de Gestión (ISO 14001, ISO 9001 e ISO 45001).
\end{cvitems}}
    \cventry{Gestión Integral HQSE}{Diploma en SSOMA}{Lima, Perú}{agosto2020-enero/2021}{\begin{cvitems}
\item Diploma en Seguridad, Salud Ocupacional y Medio Ambiente.
\end{cvitems}}
\end{cventries}

\hypertarget{habilidades}{%
\section{Habilidades}\label{habilidades}}

\hypertarget{idiomas}{%
\subsection{Idiomas}\label{idiomas}}

\begin{table}[H]
\centering\begingroup\fontsize{8}{10}\selectfont

\begin{tabular}{>{\centering\arraybackslash}p{5cm}>{\centering\arraybackslash}p{5cm}>{\centering\arraybackslash}p{5cm}}
\toprule
Skill & Spanish & English\\
\midrule
Reading & \textcolor[HTML]{17202a}{Native} & \textcolor[HTML]{17202a}{B2}\\
Writing & \textcolor[HTML]{17202a}{Native} & \textcolor[HTML]{17202a}{B2}\\
Listening & \textcolor[HTML]{17202a}{Native} & \textcolor[HTML]{17202a}{B2}\\
Speaking & \textcolor[HTML]{17202a}{Native} & \textcolor[HTML]{17202a}{B2}\\
\bottomrule
\multicolumn{3}{l}{\rule{0pt}{1em}\textit{ } El nivel B2 es considerado como un nivel intermedio que califica al quien lo posee como usuario independiente}\\
\end{tabular}
\endgroup{}
\end{table}

\hypertarget{referencias-laborales}{%
\section{Referencias laborales}\label{referencias-laborales}}

\begin{itemize}
\item
  \textbf{Blgo. Jimmy Quinaya Gutierrez}, Especialista Ambiental, cel:
  952915546,
  \href{mailto:jimmy.quinaya@unjbg.edu.pe}{\nolinkurl{jimmy.quinaya@unjbg.edu.pe}}
\item
  \textbf{Ing. Carmen Villalba Centeno}, Especialista Ambiental, cel:
  962828202
\end{itemize}


\label{LastPage}~
\end{document}
